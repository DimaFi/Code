\documentclass[a4paper,12pt]{article}
\usepackage[T2A]{fontenc}
\usepackage[english]{babel} % языковой пакет
\usepackage{graphicx} % для картинок
\usepackage{amsmath,amsfonts,amssymb} %математика
\usepackage{mathtools}
\everymath{\displaystyle}

\begin{document}
\section{Полином, Ньютона}\label{}
\subsection{Что такое полином Ньютона?}
Полином Ньютона — это способ
найти полином, который проходит через заданные точки. 
То есть, у нас есть несколько точек на графике, и нам нужно найти формулу, которая описывает линию, проходящую через эти точки. Для этого используются разделённые разности — это способ вычисления коэффициентов для полинома.


Вот основные формулы и шаги для построения полинома Ньютона, которые подойдут для конспекта:

\subsubsection{Полином Ньютона}

Полином Ньютона для набора точек \( (x_0, y_0), (x_1, y_1), \dots, (x_n, y_n) \) записывается в виде:

\[
P_n(x) = f(x_0) + \sum_{i=1}^n f[x_0, x_1, \dots, x_i] \cdot \prod_{j=0}^{i-1} (x - x_j)
\]

Где \( f[x_0, x_1, \dots, x_i] \) — разделённые разности, которые вычисляются рекурсивно.

\subsubsection{Разделенные разности}

Для вычисления разделённых разностей используется следующая рекурсивная формула:

\[
f[x_i] = y_i
\]

\[
f[x_i, x_{i+1}] = \frac{f[x_{i+1}] - f[x_i]}{x_{i+1} - x_i}
\]

\[
f[x_i, x_{i+1}, \dots, x_{i+k}] = \frac{f[x_{i+1}, \dots, x_{i+k}] - f[x_i, \dots, x_{i+k-1}]}{x_{i+k} - x_i}
\]

\subsubsection{Алгоритм вычисления полинома Ньютона}

1. Задать точки \( (x_0, y_0), (x_1, y_1), \dots, (x_n, y_n) \).
2. Вычислить разделённые разности по рекурсивной формуле.
3. Построить полином Ньютона, используя формулу:

\[
P_n(x) = f(x_0) + f[x_0, x_1](x - x_0) + f[x_0, x_1, x_2](x - x_0)(x - x_1) + \dots
\]

\subsubsection{Пример для трёх точек}

Если заданы три точки \( (x_0, y_0), (x_1, y_1), (x_2, y_2) \), полином второго порядка будет:

\[
P_2(x) = y_0 + f[x_0, x_1](x - x_0) + f[x_0, x_1, x_2](x - x_0)(x - x_1)
\]

Где:

\[
f[x_0, x_1] = \frac{y_1 - y_0}{x_1 - x_0}, \quad f[x_0, x_1, x_2] = \frac{f[x_1, x_2] - f[x_0, x_1]}{x_2 - x_0}
\]

Это основные формулы и шаги для понимания метода Ньютона для интерполяции.
\begin{enumerate}
    \item Первый
    \item Второй
    \item Третий
\end{enumerate}

\end{document}