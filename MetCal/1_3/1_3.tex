\documentclass[a4paper,12pt]{article}
\usepackage[T2A]{fontenc}
\usepackage[english]{babel} % языковой пакет
\usepackage{graphicx} % для картинок
\usepackage{amsmath,amsfonts,amssymb} %математика
\usepackage{mathtools}


\everymath{\displaystyle}

\begin{document}
\section{Полином, Ньютона}\label{}
\subsection{Что такое полином Ньютона?}
Полином Ньютона — это способ
найти полином, который проходит через заданные точки. 
То есть, у нас есть несколько точек на графике, и нам нужно найти формулу, которая описывает линию, проходящую через эти точки. Для этого используются разделённые разности — это способ вычисления коэффициентов для полинома.


Вот основные формулы и шаги для построения полинома Ньютона, которые подойдут для конспекта:

\subsubsection{Полином Ньютона}

Полином Ньютона для набора точек \( (x_0, y_0), (x_1, y_1), \dots, (x_n, y_n) \) записывается в виде:

\[
P_n(x) = f(x_0) + \sum_{i=1}^n f[x_0, x_1, \dots, x_i] \cdot \prod_{j=0}^{i-1} (x - x_j)
\]

Где \( f[x_0, x_1, \dots, x_i] \) — разделённые разности, которые вычисляются рекурсивно.

\subsubsection{Разделенные разности}

Для вычисления разделённых разностей используется следующая рекурсивная формула:

\[
f[x_i] = y_i
\]

\[
f[x_i, x_{i+1}] = \frac{f[x_{i+1}] - f[x_i]}{x_{i+1} - x_i}
\]

\[
f[x_i, x_{i+1}, \dots, x_{i+k}] = \frac{f[x_{i+1}, \dots, x_{i+k}] - f[x_i, \dots, x_{i+k-1}]}{x_{i+k} - x_i}
\]

\subsubsection{Алгоритм вычисления полинома Ньютона}

1. Задать точки \( (x_0, y_0), (x_1, y_1), \dots, (x_n, y_n) \).
2. Вычислить разделённые разности по рекурсивной формуле.
3. Построить полином Ньютона, используя формулу:

\[
P_n(x) = f(x_0) + f[x_0, x_1](x - x_0) + f[x_0, x_1, x_2](x - x_0)(x - x_1) + \dots
\]

\subsubsection{Пример для трёх точек}

Если заданы три точки \( (x_0, y_0), (x_1, y_1), (x_2, y_2) \), полином второго порядка будет:

\[
P_2(x) = y_0 + f[x_0, x_1](x - x_0) + f[x_0, x_1, x_2](x - x_0)(x - x_1)
\]

Где:

\[
f[x_0, x_1] = \frac{y_1 - y_0}{x_1 - x_0}, \quad f[x_0, x_1, x_2] = \frac{f[x_1, x_2] - f[x_0, x_1]}{x_2 - x_0}
\]

Это основные формулы и шаги для понимания метода Ньютона для интерполяции.


\section*{Интерполяция методом разделённых разностей Ньютона}

Для построения интерполяционного многочлена Ньютона необходимо использовать разделённые разности, которые позволяют поэтапно строить многочлен для заданного набора точек.

\subsection*{Исходные данные}

Допустим, у нас есть следующие точки:
\[
(x_0, y_0) = (0, 12), \quad (x_1, y_1) = (1, 13), \quad (x_2, y_2) = (2, 20)
\]

Цель — построить интерполяционный многочлен Ньютона, используя разделённые разности, и затем вычислить значения многочлена в промежуточных точках, таких как $x = 0.5$, $x = 1.5$ и $x = 2.5$.

\subsection*{Вычисление разделённых разностей}

Для начала вычислим разделённые разности.

\subsubsection*{Первая разделённая разность}

Первая разделённая разность между точками $x_0$ и $x_1$ вычисляется по формуле:
\[
f[x_0, x_1] = \frac{y_1 - y_0}{x_1 - x_0} = \frac{13 - 12}{1 - 0} = 1.0
\]
Аналогично для точек $x_1$ и $x_2$:
\[
f[x_1, x_2] = \frac{y_2 - y_1}{x_2 - x_1} = \frac{20 - 13}{2 - 1} = 7.0
\]

\subsubsection*{Вторая разделённая разность}

Для вычисления второй разделённой разности между точками $x_0$, $x_1$ и $x_2$, используем формулу:
\[
f[x_0, x_1, x_2] = \frac{f[x_1, x_2] - f[x_0, x_1]}{x_2 - x_0} = \frac{7.0 - 1.0}{2 - 0} = \frac{6.0}{2} = 3.0
\]

Таким образом, разделённые разности следующие:
\[
f[x_0, x_1] = 1.0, \quad f[x_1, x_2] = 7.0, \quad f[x_0, x_1, x_2] = 3.0
\]

\subsection*{Построение интерполяционного многочлена Ньютона}

Интерполяционный многочлен Ньютона второго порядка для трёх точек $(x_0, y_0)$, $(x_1, y_1)$ и $(x_2, y_2)$ имеет вид:
\[
P_2(x) = y_0 + f[x_0, x_1] \cdot (x - x_0) + f[x_0, x_1, x_2] \cdot (x - x_0)(x - x_1)
\]
Подставим известные значения:
\[
P_2(x) = 12 + 1.0 \cdot (x - 0) + 3.0 \cdot (x - 0)(x - 1)
\]
Упростим выражение:
\[
P_2(x) = 12 + (x - 0) + 3.0 \cdot (x - 0)(x - 1)
\]
\[
P_2(x) = 12 + x + 3.0 \cdot x(x - 1)
\]

\subsection*{Вычисление значений многочлена}

Теперь можем вычислить значения многочлена $P_2(x)$ в точках $x = 0.5$, $x = 1.5$ и $x = 2.5$.

\subsubsection*{Вычисление $P_2(0.5)$}

Подставим $x = 0.5$ в многочлен:
\[
P_2(0.5) = 12 + 0.5 + 3.0 \cdot 0.5 \cdot (0.5 - 1)
\]
\[
P_2(0.5) = 12 + 0.5 + 3.0 \cdot 0.5 \cdot (-0.5)
\]
\[
P_2(0.5) = 12 + 0.5 - 0.75 = 11.75
\]

\subsubsection*{Вычисление $P_2(1.5)$}

Подставим $x = 1.5$ в многочлен:
\[
P_2(1.5) = 12 + 1.5 + 3.0 \cdot 1.5 \cdot (1.5 - 1)
\]
\[
P_2(1.5) = 12 + 1.5 + 3.0 \cdot 1.5 \cdot 0.5
\]
\[
P_2(1.5) = 12 + 1.5 + 2.25 = 15.75
\]

\subsubsection*{Вычисление $P_2(2.5)$}

Подставим $x = 2.5$ в многочлен:
\[
P_2(2.5) = 12 + 2.5 + 3.0 \cdot 2.5 \cdot (2.5 - 1)
\]
\[
P_2(2.5) = 12 + 2.5 + 3.0 \cdot 2.5 \cdot 1.5
\]
\[
P_2(2.5) = 12 + 2.5 + 11.25 = 25.75
\]

\subsection*{Заключение}

Таким образом, значения многочлена Ньютона во введённых точках:
\[
P_2(0.5) = 11.75, \quad P_2(1.5) = 15.75, \quad P_2(2.5) = 25.75
\]
Эти значения были вычислены с использованием метода разделённых разностей и построения интерполяционного многочлена Ньютона.
\end{document}
