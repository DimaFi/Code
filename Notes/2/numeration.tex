\documentclass[12pt]{article} % Увеличенный размер шрифта
\usepackage[utf8]{inputenc}
\usepackage[russian]{babel}
\usepackage{geometry} % Для изменения полей
\usepackage{setspace} % Для настройки междустрочного интервала

% Настройки полей страницы
\geometry{
    a4paper,
    left=25mm,
    right=25mm,
    top=25mm,
    bottom=25mm
}

% Междустрочный интервал
\onehalfspacing % Полуторный интервал (можно заменить на \doublespacing для двойного)

\title{Китайская нумерация}
\author{}
\date{}

\begin{document}

\maketitle

\section*{Китайская нумерация}

Китайская нумерация является одной из старейших и в то же время прогрессивных систем счисления. Основные принципы этой системы схожи с принципами современной арабской нумерации, которой мы пользуемся сегодня. Возникла она около 4000 лет назад в Китае.

\textbf{Система счисления} — это сложное понятие, включающее в себя правила записи, чтения и выполнения операций с числами. В древности китайские цифры записывались с больших разрядов к меньшим. Если в числе отсутствовал какой-либо разряд, будь то десятки, единицы или другой, его просто пропускали, переходя к следующему разряду. Позже был введён специальный знак для обозначения пустого разряда — кружок, который стал аналогом нуля. Чтобы избежать путаницы с разрядами, применялись служебные иероглифы, которые писались после основного символа и показывали значение цифры в конкретном разряде.

Примером записи числа 548 в китайской нумерации является его \textbf{мультипликативная} структура, где числа записываются через умножение: 
\[
1 \times 1 000 + 5 \times 100 + 4 \times 10 + 8
\]

Если один из символов, обозначающих числа от 1 до 9, стоит перед иероглифом, обозначающим степень числа 10, его нужно умножить на соответствующую степень. Если же символ стоит на последнем месте, то его необходимо просто прибавить к предыдущим значениям.

Таким образом, первоначально китайская система счёта была мультипликативной, поскольку использовала умножение как ключевой принцип для обозначения чисел.

\end{document}
