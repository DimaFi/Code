\documentclass[a4paper,12pt]{article}
\usepackage[utf8]{inputenc}
\usepackage[russian]{babel}
\usepackage{amsmath}
\usepackage{amsfonts}
\usepackage{amssymb}
\usepackage{geometry}
\geometry{top=2cm,bottom=2cm,left=3cm,right=1.5cm}

\begin{document}

\title{Криптоанализ и методы мониторинга трафика в протоколах прокси: исследование на примере Shadowsocks}
\date{\today}
\maketitle

\section*{Тема:}

Криптоанализ и методы мониторинга трафика в протоколах прокси: исследование на примере Shadowsocks.

\section*{План (цель и задачи):}

\begin{enumerate}
    \item Изучить уязвимости прокси-сервисов.
    \item Провести криптоанализ с использованием известных атак.
    \item Проанализировать методы мониторинга и блокировки трафика.
\end{enumerate}

\section*{Актуальность}

В эпоху быстрого роста объемов интернет-трафика и роста угроз информационной безопасности изучение шифрования и методов анализа зашифрованного трафика стало критически важным. Прокси-протоколы, такие как Shadowsocks, используются для обхода интернет-цензуры и защиты данных пользователей, что порождает интерес со стороны исследователей и разработчиков систем мониторинга. Таким образом, исследование возможностей криптоанализа и методов мониторинга является актуальной задачей для обеспечения безопасности сетей и контроля трафика.

\newpage

\section*{Теоретическая часть}
\textbf{Криптоанализ и методы мониторинга трафика в протоколах прокси: исследование на примере Shadowsocks}

\subsection*{1. Протоколы передачи данных и работа прокси-серверов}

Прокси-серверы действуют как посредники между клиентом и сервером, обеспечивая различные функции, включая шифрование трафика, сокрытие IP-адресов и обход интернет-цензуры. Например, Shadowsocks использует \textit{SOCKS5-прокси}, который позволяет передавать любые виды данных (например, HTTP, FTP) через зашифрованный канал. Это универсальный инструмент для обеспечения приватности и обхода блокировок, но при этом шифрование может подвергаться анализу с целью обнаружения трафика прокси.

\subsection*{2. Основы криптографии и их использование в Shadowsocks}

Shadowsocks использует симметричные алгоритмы шифрования, такие как \textbf{AES} и \textbf{ChaCha20}, для защиты данных. 

\begin{itemize}
    \item \textbf{AES (Advanced Encryption Standard)} --- наиболее популярный симметричный шифр, широко применяемый благодаря своей высокой производительности и безопасности. В Shadowsocks используются несколько режимов AES, таких как CBC (Cipher Block Chaining) и CTR (Counter Mode).
    \item \textbf{ChaCha20} --- потоковый шифр, известный своей эффективностью на мобильных устройствах и платформах с ограниченными ресурсами. Его скорость и безопасность делают его отличной альтернативой AES для защищённой передачи данных.
\end{itemize}

Понимание этих алгоритмов и их уязвимостей необходимо для оценки безопасности трафика, передаваемого через Shadowsocks.

\subsection*{3. Криптоанализ и методы мониторинга трафика}

Криптоанализ заключается в попытке выявления уязвимостей шифрования. В случае Shadowsocks это может включать анализ трафика на предмет выявления характерных признаков прокси-трафика или использование атак на шифровальные алгоритмы.

Методы криптоанализа включают:

\begin{itemize}
    \item \textbf{Анализ трафика}: Изучение характеристик трафика, таких как размер и частота пакетов, позволяет выявить использование зашифрованных прокси, таких как Shadowsocks. Особое внимание уделяется времени передачи данных и особенностям их распределения.
    \item \textbf{Атаки на шифры}: Классические методы криптоанализа, такие как атаки по времени (timing attacks) или атаки через сторонние каналы (side-channel attacks), могут быть применимы к прокси-сервисам.
\end{itemize}

Также следует учитывать методы мониторинга, такие как \textbf{глубокий анализ пакетов (DPI)}, который применяется для обнаружения и блокировки зашифрованного трафика на уровне сетевого провайдера.

\subsection*{4. Методы блокировки и мониторинга прокси-сервисов}

Для мониторинга и блокировки прокси-сервисов применяются следующие методы:

\begin{itemize}
    \item \textbf{Глубокий анализ пакетов (DPI)} --- метод анализа сетевого трафика, который позволяет провайдерам интернета и государственным органам выявлять зашифрованный прокси-трафик. DPI может анализировать содержимое пакетов и выявлять характерные признаки использования прокси-сервисов.
    \item \textbf{Фильтрация по IP-адресам и DNS}: Поскольку Shadowsocks использует определённые IP-адреса и DNS-сервера для проксирования трафика, блокировка этих адресов позволяет эффективно ограничивать использование прокси.
    \item \textbf{Маскировка трафика}: Shadowsocks и подобные протоколы внедряют методы обфускации, чтобы их трафик был неотличим от обычного HTTPS-трафика, что затрудняет его обнаружение.
\end{itemize}

Анализ этих методов позволяет глубже понять, как работают системы мониторинга трафика и какие техники можно использовать для защиты сетей.

\newpage

\section*{Обзор источников}

\begin{itemize}
    \item Статья *"ACER: Detecting Shadowsocks Server Based on Active Probe Technology"* обсуждает методы активного зондирования для выявления Shadowsocks-серверов и может быть полезна для понимания уязвимостей протокола. Важно изучить, как изменяется поведение сервера при обработке подозрительных запросов, что помогает разработчикам систем мониторинга трафика8203;
    \item В исследовании *"Detecting Probe-resistant Proxies"* (2020) рассматриваются методы анализа слабостей серверов прокси при ответах на недействительные запросы, что помогает детектировать использование Shadowsocks в сетях с высокими объемами трафика. Авторы также обсуждают способы обхода систем анализа данных, что является актуальной темой для тех, кто изучает мониторинг прокси-сервисов
    \item Статья *"Deep Learning for Encrypted Traffic Classification"* (2019) подробно рассматривает использование методов глубокого обучения для анализа зашифрованного трафика, что особенно важно при анализе и классификации трафика, передаваемого через такие протоколы, как VPN и Shadowsocks
    \item Книга *"Cryptography and Network Security: Principles and Practice"* (William Stallings, 2017) предоставляет фундаментальные знания по криптографии и сетевой безопасности, которые будут полезны для проведения криптоанализа Shadowsocks и других прокси-протоколов.
\end{itemize}

\end{document}
