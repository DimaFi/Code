\documentclass[a4paper,12pt]{article}
\usepackage{amsmath, amssymb, amsthm}
\usepackage{pgfplots}
\pgfplotsset{compat=1.16}
\usepackage{geometry}
\usepackage[english,russian]{babel}

\geometry{top=2cm, bottom=2cm, left=2cm, right=2cm}

\begin{document}

\section*{Решение заданий}

\subsection*{1. Построение линий уровня функции \( z = |x| + |y|, \, z > 0 \)}

Функция \( z = |x| + |y| \) является кусочно-линейной, а её линии уровня имеют вид ромбов с вершинами на осях координат. Линии уровня соответствуют уравнениям:
\[
|x| + |y| = C, \quad C > 0.
\]
    
\begin{center}
\begin{tikzpicture}
    \begin{axis}[
        axis equal,
        xlabel={$x$}, ylabel={$y$},
        title={Линии уровня $z = |x| + |y|$},
        grid=both, xmin=-4, xmax=4, ymin=-4, ymax=4]
        \foreach \z in {1, 2, 3} {
            \addplot[domain=-\z:\z, samples=500, color=blue] ({\z - abs(x)}, x);
            \addplot[domain=-\z:\z, samples=500, color=blue] ({- (\z - abs(x))}, x);
        }
    \end{axis}
\end{tikzpicture}
\end{center}

На графике показаны линии уровня \( z = 1 \), \( z = 2 \), \( z = 3 \).

\subsection*{2. Вычисление градиента и матрицы Гессе для \( z = \ln(2x^2 - y) \)}

Функция:
\[
z(x, y) = \ln(2x^2 - y), \quad \text{где } 2x^2 - y > 0.
\]

\paragraph{Градиент:}
\[
\frac{\partial z}{\partial x} = \frac{4x}{2x^2 - y}, \quad
\frac{\partial z}{\partial y} = -\frac{1}{2x^2 - y}.
\]
\[
\nabla z = \left( \frac{4x}{2x^2 - y}, -\frac{1}{2x^2 - y} \right).
\]

\paragraph{Гессиан:}
\[
\frac{\partial^2 z}{\partial x^2} = \frac{4(2x^2 - y) - 16x^2}{(2x^2 - y)^2} = \frac{-8x^2 - 4y}{(2x^2 - y)^2},
\]
\[
\frac{\partial^2 z}{\partial x \partial y} = \frac{4x}{(2x^2 - y)^2}, \quad
\frac{\partial^2 z}{\partial y^2} = -\frac{1}{(2x^2 - y)^2}.
\]
Матрица Гессе:
\[
H(z) =
\begin{bmatrix}
\frac{-8x^2 - 4y}{(2x^2 - y)^2} & \frac{4x}{(2x^2 - y)^2} \\
\frac{4x}{(2x^2 - y)^2} & -\frac{1}{(2x^2 - y)^2}
\end{bmatrix}.
\]

\paragraph{Проверка в точке \( C(1, 1) \):}
\[
\nabla z = \left( \frac{4 \cdot 1}{2 \cdot 1^2 - 1}, -\frac{1}{2 \cdot 1^2 - 1} \right) = (4, -1).
\]
\[
H(z) =
\begin{bmatrix}
\frac{-8 \cdot 1^2 - 4 \cdot 1}{(2 \cdot 1^2 - 1)^2} & \frac{4 \cdot 1}{(2 \cdot 1^2 - 1)^2} \\
\frac{4 \cdot 1}{(2 \cdot 1^2 - 1)^2} & -\frac{1}{(2 \cdot 1^2 - 1)^2}
\end{bmatrix} =
\begin{bmatrix}
-12 & 4 \\
4 & -1
\end{bmatrix}.
\]

\subsection*{3. Выпуклость функции \( f(x) = x_1^2 + x_2^4 \)}

Гессиан функции:
\[
\frac{\partial^2 f}{\partial x_1^2} = 2, \quad
\frac{\partial^2 f}{\partial x_2^2} = 12x_2^2, \quad
\frac{\partial^2 f}{\partial x_1 \partial x_2} = 0.
\]
Матрица Гессе:
\[
H(f) =
\begin{bmatrix}
2 & 0 \\
0 & 12x_2^2
\end{bmatrix}.
\]
Так как собственные значения \( \lambda_1 = 2 \), \( \lambda_2 = 12x_2^2 \geq 0 \), матрица \( H(f) \) положительно определена. Следовательно, функция \( f(x) \) выпуклая.

\subsection*{4. Экстремумы функции \( f(x) = x_1^2 x_2^2 (1 - x_1 - x_2) \)}

\paragraph{Градиент:}
\[
\frac{\partial f}{\partial x_1} = 2x_1 x_2^2 (1 - x_1 - x_2) - x_1^2 x_2^2, \quad
\frac{\partial f}{\partial x_2} = 2x_2 x_1^2 (1 - x_1 - x_2) - x_1^2 x_2^2.
\]

Решая систему \(\nabla f = 0\), получаем критические точки:
\[
(0, 0), \quad (0, 1), \quad \left(\frac{2}{5}, \frac{2}{5}\right), \quad \left(\frac{2}{3}, 0\right).
\]

\paragraph{Гессиан:}
\[
H(f) =
\begin{bmatrix}
\frac{\partial^2 f}{\partial x_1^2} & \frac{\partial^2 f}{\partial x_1 \partial x_2} \\
\frac{\partial^2 f}{\partial x_2 \partial x_1} & \frac{\partial^2 f}{\partial x_2^2}
\end{bmatrix}.
\]
После подстановки точек:
- В точке \( \left(\frac{2}{5}, \frac{2}{5}\right) \): локальный максимум.
- В точке \( \left(\frac{2}{3}, 0\right) \): локальный минимум.

\paragraph{График:}
\begin{center}
\begin{tikzpicture}
    \begin{axis}[
        view={60}{30},
        xlabel={$x_1$}, ylabel={$x_2$}, zlabel={$f(x)$},
        title={График функции $f(x)$},
        colormap/viridis]
        \addplot3[surf, domain=0:1, domain y=0:1, samples=30]
            {x^2 * y^2 * (1 - x - y)};
    \end{axis}
\end{tikzpicture}
\end{center}

\end{document}
