\documentclass[a4paper,12pt]{article}
\usepackage[utf8]{inputenc}
\usepackage{amsmath, amssymb}
\usepackage{pgfplots}  % Для построения графиков
\pgfplotsset{compat=1.17}
\usepackage{geometry}
\usepackage[english,russian]{babel}
\geometry{top=2cm, bottom=2cm, left=2cm, right=2cm}

\title{Лабораторная 4}
\author{Филиппенко Дмитрий Александрович, 341 группа}
\date{Вариант 24}           

\begin{document}

\maketitle

\section*{Задание 1: Построение линий уровня}
Дана функция:
\[
z = |x| + |y|, \quad z > 0.
\]
Линии уровня представляют собой ромбы, заданные уравнением:
\[
|x| + |y| = C, \quad C > 0.
\]
График линий уровня для нескольких значений \( C \):
\begin{center}
\begin{tikzpicture}
    \begin{axis}[
        axis equal,
        xlabel={$x$},
        ylabel={$y$},
        grid=major,
        title={Линии уровня $z = |x| + |y|$}
    ]
        % Построение линий уровня
        \addplot[domain=-2:2, samples=200, thick, blue] ({x}, {2-abs(x)});
        \addplot[domain=-2:2, samples=200, thick, blue] ({x}, {1-abs(x)});
        \addplot[domain=-2:2, samples=200, thick, blue] ({x}, {0.5-abs(x)});
        \addplot[domain=-2:2, samples=200, thick, blue] ({x}, {3-abs(x)});
    \end{axis}
\end{tikzpicture}
\end{center}

\section*{Задание 2: Градиент и матрица Гессе}
Дана функция:
\[
z = \ln(2x^2 - y), \quad (2x^2 - y > 0).
\]

\subsection*{Градиент:}
\[
\nabla z = \left( \frac{\partial z}{\partial x}, \frac{\partial z}{\partial y} \right) = 
\left( \frac{4x}{2x^2 - y}, -\frac{1}{2x^2 - y} \right).
\]

\subsection*{Матрица Гессе:}
\[
H(z) =
\begin{bmatrix}
\frac{-8x^2 - 4y}{(2x^2 - y)^2} & \frac{4x}{(2x^2 - y)^2} \\
\frac{4x}{(2x^2 - y)^2} & -\frac{1}{(2x^2 - y)^2}
\end{bmatrix}.
\]

\subsection*{Результаты в точках:}
1. В точке \( A(1, 2) \): функция не определена, так как \( 2x^2 - y = 0 \).
2. В точке \( B(2, 1) \):
\[
\nabla z = \left( \frac{8}{7}, -\frac{1}{7} \right), \quad
H(z) =
\begin{bmatrix}
-\frac{68}{49} & \frac{8}{49} \\
\frac{8}{49} & -\frac{1}{49}
\end{bmatrix}.
\]
3. В точке \( C(1, 1) \):
\[
\nabla z = (4, -1), \quad
H(z) =
\begin{bmatrix}
-12 & 4 \\
4 & -1
\end{bmatrix}.
\]

\section*{Задание 3: Выпуклость функции}
Дана функция:
\[
f(x) = x_1^2 + x_2^4, \quad x \in E_2.
\]
Матрица Гессе:
\[
H(f) =
\begin{bmatrix}
2 & 0 \\
0 & 12x_2^2
\end{bmatrix}.
\]
Собственные значения матрицы:
\[
\lambda_1 = 2 > 0, \quad \lambda_2 = 12x_2^2 \geq 0.
\]
Следовательно, \( f(x) \) — выпуклая функция.

\section*{Задание 4: Нахождение экстремумов}
Дана функция:
\[
f(x) = x_1^2 x_2^2 (1 - x_1 - x_2), \quad x \in E_2.
\]

\subsection*{Критические точки:}
\[
\nabla f = 0 \quad \Rightarrow \quad 
(x_1, x_2) \in \{ (0, 0), (0, 1), (1, 0), (\frac{2}{5}, \frac{2}{5}) \}.
\]

\subsection*{Анализ типа экстремумов:}
- В точке \( (0, 0) \): \( f(0, 0) = 0 \) (глобальный минимум, так как \( f(x) \geq 0 \)).
- В точке \( (\frac{2}{5}, \frac{2}{5}) \): локальный максимум \( f(\frac{2}{5}, \frac{2}{5}) = \frac{16}{625} \).
- На границах области: анализируем поведение \( f(x_1, x_2) \to 0 \).

\subsection*{График функции:}
\begin{center}
\begin{tikzpicture}
    \begin{axis}[
        xlabel={$x_1$},
        ylabel={$x_2$},
        zlabel={$f(x)$},
        view={60}{30},
        grid=major,
        title={График функции $f(x_1, x_2)$}
    ]
        \addplot3[surf, domain=0:1, y domain=0:1, samples=30] {x^2 * y^2 * (1-x-y)};
    \end{axis}
\end{tikzpicture}
\end{center}

\end{document}
