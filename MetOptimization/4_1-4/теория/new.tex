\documentclass[a4paper,12pt]{article}
\usepackage[utf8]{inputenc}
\usepackage{amsmath, amssymb}
\usepackage{pgfplots}  % Для построения графиков
\pgfplotsset{compat=1.17}
\usepackage{geometry}
\usepackage[english,russian]{babel}
\geometry{top=2cm, bottom=2cm, left=2cm, right=2cm}

\title{Лабораторная 4}
\author{Филиппенко Дмитрий Александрович, 341 группа}
\date{Вариант 24}           

\begin{document}

\maketitle

\section*{Необходимая теория}

Для выполнения заданий, связанных с анализом функций многих переменных, необходимо знание следующих теоретических основ:

\subsection*{1. Линии уровня функции}

\textbf{Линии уровня} функции \( f(x, y) \) — это множество точек \( (x, y) \), для которых функция принимает фиксированное значение \( C \), то есть:
\[
f(x, y) = C.
\]
Это уравнение описывает "горизонтальные" срезы графика функции. Например, для функции \( z = f(x, y) \), линия уровня для значения \( z = C \) будет множеством точек на плоскости \( x, y \), где функция \( f(x, y) \) равна \( C \).

\begin{itemize}
    \item Линии уровня показывают, как функция ведет себя на плоскости \( x, y \) для различных значений \( C \).
    \item Для функций, зависимых от двух переменных, линии уровня могут быть кривыми на плоскости \( xy \), и форма этих кривых зависит от свойств самой функции.
    \item Линии уровня полезны для визуализации поведения функции.
\end{itemize}

\subsection*{2. Градиент функции}

\textbf{Градиент} функции \( f(x, y) \) — это вектор, который указывает направление наибольшего возрастания функции в каждой точке:
\[
\nabla f(x, y) = \left( \frac{\partial f}{\partial x}, \frac{\partial f}{\partial y} \right).
\]
Градиент включает в себя частные производные функции по каждой переменной. Он направлен в сторону наибольшего увеличения функции.

\begin{itemize}
    \item Если \( \nabla f(x_0, y_0) = (0, 0) \), то точка \( (x_0, y_0) \) может быть критической точкой.
    \item Градиент перпендикулярен линиям уровня функции. Это означает, что вектор градиента направлен под углом 90° к линиям уровня в точке.
\end{itemize}

\subsection*{3. Матрица Гессе}

\textbf{Матрица Гессе} функции \( f(x, y) \) — это матрица, состоящая из вторых частных производных функции:
\[
H(f) =
\begin{bmatrix}
\frac{\partial^2 f}{\partial x^2} & \frac{\partial^2 f}{\partial x \partial y} \\
\frac{\partial^2 f}{\partial y \partial x} & \frac{\partial^2 f}{\partial y^2}
\end{bmatrix}.
\]
Матрица Гессе позволяет анализировать кривизну функции и помогает определить, является ли критическая точка точкой минимума, максимума или седловой точкой.

\begin{itemize}
    \item Если матрица Гессе положительно определена (все её собственные значения положительны), то функция имеет локальный минимум в данной точке.
    \item Если матрица Гессе отрицательно определена (все её собственные значения отрицательны), то функция имеет локальный максимум в данной точке.
    \item Если матрица Гессе неопределённая (собственные значения имеют разные знаки), то точка является седловой.
    \item Матрица Гессе симметрична, что облегчает её анализ.
\end{itemize}

\subsection*{4. Критические точки и экстремумы}

\textbf{Критическая точка} функции — это точка, в которой градиент функции равен нулю:
\[
\nabla f(x_0, y_0) = (0, 0).
\]
Чтобы найти критические точки, нужно решить систему уравнений для частных производных функции, приравняв их к нулю:
\[
\frac{\partial f}{\partial x} = 0, \quad \frac{\partial f}{\partial y} = 0.
\]
После нахождения критических точек необходимо определить, является ли эта точка минимумом, максимумом или седловой точкой. Это можно сделать, используя матрицу Гессе и анализируя её собственные значения.

\begin{itemize}
    \item Если все собственные значения матрицы Гессе положительны, то критическая точка является локальным минимумом.
    \item Если все собственные значения отрицательны, то критическая точка является локальным максимумом.
    \item Если собственные значения матрицы Гессе имеют разные знаки, то критическая точка является седловой.
\end{itemize}

\subsection*{5. Выпуклость функций}

Функция \( f(x) \) называется \textbf{выпуклой}, если для любых двух точек \( x_1, x_2 \) области определения функции выполнено неравенство:
\[
f(\alpha x_1 + (1-\alpha)x_2) \leq \alpha f(x_1) + (1-\alpha) f(x_2), \quad \alpha \in [0, 1].
\]
Это означает, что график функции всегда лежит ниже прямой, соединяющей любые две её точки.

\begin{itemize}
    \item Если функция выпуклая, то её локальный минимум является глобальным минимумом.
    \item Если функция строго выпуклая, то локальный минимум единственен.
    \item Выпуклость функции можно проверить с помощью матрицы Гессе: если её собственные значения неотрицательны, то функция выпуклая.
    \item Если все собственные значения матрицы Гессе положительны, то функция строго выпуклая.
\end{itemize}

\subsection*{6. Алгоритм нахождения экстремумов}

1. Найдите \textbf{критические точки}, решив систему уравнений \( \nabla f = 0 \).
2. Построите \textbf{матрицу Гессе} для каждой критической точки.
3. Проанализируйте матрицу Гессе:
    \begin{itemize}
        \item Если все собственные значения матрицы Гессе положительны, то точка является локальным минимумом.
        \item Если все собственные значения матрицы Гессе отрицательны, то точка является локальным максимумом.
        \item Если собственные значения матрицы Гессе имеют разные знаки, то точка является седловой.
    \end{itemize}\section*{Необходимая теория}

Для выполнения заданий, связанных с анализом функций многих переменных, необходимо знание следующих теоретических основ:

\subsection*{1. Линии уровня функции}

\textbf{Линии уровня} функции \( f(x, y) \) — это множество точек \( (x, y) \), для которых функция принимает фиксированное значение \( C \), то есть:
\[
f(x, y) = C.
\]
Это уравнение описывает "горизонтальные" срезы графика функции. Например, для функции \( z = f(x, y) \), линия уровня для значения \( z = C \) будет множеством точек на плоскости \( x, y \), где функция \( f(x, y) \) равна \( C \).

\begin{itemize}
    \item Линии уровня показывают, как функция ведет себя на плоскости \( x, y \) для различных значений \( C \).
    \item Для функций, зависимых от двух переменных, линии уровня могут быть кривыми на плоскости \( xy \), и форма этих кривых зависит от свойств самой функции.
    \item Линии уровня полезны для визуализации поведения функции.
\end{itemize}

\subsection*{2. Градиент функции}

\textbf{Градиент} функции \( f(x, y) \) — это вектор, который указывает направление наибольшего возрастания функции в каждой точке:
\[
\nabla f(x, y) = \left( \frac{\partial f}{\partial x}, \frac{\partial f}{\partial y} \right).
\]
Градиент включает в себя частные производные функции по каждой переменной. Он направлен в сторону наибольшего увеличения функции.

\begin{itemize}
    \item Если \( \nabla f(x_0, y_0) = (0, 0) \), то точка \( (x_0, y_0) \) может быть критической точкой.
    \item Градиент перпендикулярен линиям уровня функции. Это означает, что вектор градиента направлен под углом 90° к линиям уровня в точке.
\end{itemize}

\subsection*{3. Матрица Гессе}

\textbf{Матрица Гессе} функции \( f(x, y) \) — это матрица, состоящая из вторых частных производных функции:
\[
H(f) =
\begin{bmatrix}
\frac{\partial^2 f}{\partial x^2} & \frac{\partial^2 f}{\partial x \partial y} \\
\frac{\partial^2 f}{\partial y \partial x} & \frac{\partial^2 f}{\partial y^2}
\end{bmatrix}.
\]
Матрица Гессе позволяет анализировать кривизну функции и помогает определить, является ли критическая точка точкой минимума, максимума или седловой точкой.

\begin{itemize}
    \item Если матрица Гессе положительно определена (все её собственные значения положительны), то функция имеет локальный минимум в данной точке.
    \item Если матрица Гессе отрицательно определена (все её собственные значения отрицательны), то функция имеет локальный максимум в данной точке.
    \item Если матрица Гессе неопределённая (собственные значения имеют разные знаки), то точка является седловой.
    \item Матрица Гессе симметрична, что облегчает её анализ.
\end{itemize}

\subsection*{4. Критические точки и экстремумы}

\textbf{Критическая точка} функции — это точка, в которой градиент функции равен нулю:
\[
\nabla f(x_0, y_0) = (0, 0).
\]
Чтобы найти критические точки, нужно решить систему уравнений для частных производных функции, приравняв их к нулю:
\[
\frac{\partial f}{\partial x} = 0, \quad \frac{\partial f}{\partial y} = 0.
\]
После нахождения критических точек необходимо определить, является ли эта точка минимумом, максимумом или седловой точкой. Это можно сделать, используя матрицу Гессе и анализируя её собственные значения.

\begin{itemize}
    \item Если все собственные значения матрицы Гессе положительны, то критическая точка является локальным минимумом.
    \item Если все собственные значения отрицательны, то критическая точка является локальным максимумом.
    \item Если собственные значения матрицы Гессе имеют разные знаки, то критическая точка является седловой.
\end{itemize}

\subsection*{5. Выпуклость функций}

Функция \( f(x) \) называется \textbf{выпуклой}, если для любых двух точек \( x_1, x_2 \) области определения функции выполнено неравенство:
\[
f(\alpha x_1 + (1-\alpha)x_2) \leq \alpha f(x_1) + (1-\alpha) f(x_2), \quad \alpha \in [0, 1].
\]
Это означает, что график функции всегда лежит ниже прямой, соединяющей любые две её точки.

\begin{itemize}
    \item Если функция выпуклая, то её локальный минимум является глобальным минимумом.
    \item Если функция строго выпуклая, то локальный минимум единственен.
    \item Выпуклость функции можно проверить с помощью матрицы Гессе: если её собственные значения неотрицательны, то функция выпуклая.
    \item Если все собственные значения матрицы Гессе положительны, то функция строго выпуклая.
\end{itemize}

\subsection*{6. Алгоритм нахождения экстремумов}

1. Найдите \textbf{критические точки}, решив систему уравнений \( \nabla f = 0 \).
2. Построите \textbf{матрицу Гессе} для каждой критической точки.
3. Проанализируйте матрицу Гессе:
    \begin{itemize}
        \item Если все собственные значения матрицы Гессе положительны, то точка является локальным минимумом.
        \item Если все собственные значения матрицы Гессе отрицательны, то точка является локальным максимумом.
        \item Если собственные значения матрицы Гессе имеют разные знаки, то точка является седловой.
    \end{itemize}
4. Если требуется определить глобальный экстремум, сравните значения функции в локальных экстремумах с учётом ограничений на области определения функции.

\subsection*{7. Линейная и нелинейная оптимизация}

\textbf{Линейная оптимизация} — это задача, в которой целевая функция и ограничения являются линейными. Для решения таких задач используются методы, такие как симплекс-метод и методы градиентного спуска.

\textbf{Нелинейная оптимизация} — это задачи с нелинейными целевыми функциями и ограничениями. Эти задачи требуют более сложных методов, таких как методы численного поиска экстремума (градиентный спуск, метод Ньютона).

\end{itemize}

4. Если требуется определить глобальный экстремум, сравните значения функции в локальных экстремумах с учётом ограничений на области определения функции.

\subsection*{7. Линейная и нелинейная оптимизация}

\textbf{Линейная оптимизация} — это задача, в которой целевая функция и ограничения являются линейными. Для решения таких задач используются методы, такие как симплекс-метод и методы градиентного спуска.

\textbf{Нелинейная оптимизация} — это задачи с нелинейными целевыми функциями и ограничениями. Эти задачи требуют более сложных методов, таких как методы численного поиска экстремума (градиентный спуск, метод Ньютона).

\end{itemize}


\end{document}
