\documentclass[a4paper,12pt]{article}
\usepackage[T2A]{fontenc}
\usepackage{amsmath}
\usepackage[english,russian]{babel}
\usepackage{graphicx}
\usepackage{geometry}
\geometry{margin=1in}
\usepackage{listings}

\title{Лабораторная 3}
\author{Филиппенко Дмитрий Александрович, 341 группа}
\date{Вариант 24}       

\begin{document}

\maketitle

\section{Задача 1: Минимизация функции методом средней точки, методом хорд и методом Ньютона}

\textbf{Функция:}
\[
f(x) = \cos(5x) - \sin(2x), \quad x \in [0, 1]
\]

Цель задачи — найти минимум данной функции на отрезке с помощью следующих методов: метода средней точки, метода хорд и метода Ньютона.

\subsection{Метод средней точки}

Метод средней точки заключается в итеративном сужении интервала, путем выбора средней точки и определения, в каком подинтервале искать минимум на основе значений производной.

\textbf{Алгоритм:}
\begin{enumerate}
    \item Инициализация интервала \( [a, b] \) и заданной точности.
    \item Вычисление производной функции в средней точке.
    \item В зависимости от знака производной сужаем интервал.
    \item Повторяем итерации до достижения требуемой точности.
\end{enumerate}

\textbf{Результат:}
\[ x_{\text{min}} \approx 0.649 \ldots, \quad f(x_{\text{min}}) \approx -1.957 \ldots \]

\subsection{Метод хорд}

Метод хорд использует секущую линию между двумя точками интервала для приближенного нахождения корня производной функции и нахождения минимума.

\textbf{Алгоритм:}
\begin{enumerate}
    \item Инициализация интервала \( [a, b] \) и заданной точности.
    \item На каждой итерации вычисляем точки пересечения секущей линии с осью \( x \).
    \item Выбираем подинтервал для следующей итерации, основываясь на значениях производной.
    \item Повторяем итерации до достижения требуемой точности.
\end{enumerate}

\textbf{Результат:}
\[ x_{\text{min}} \approx 0.649 \ldots, \quad f(x_{\text{min}}) \approx -1.957 \ldots \]

\subsection{Метод Ньютона}

Метод Ньютона использует первую и вторую производные функции для нахождения минимума путем итеративного уточнения значения на основе формулы Ньютона-Рафсона.

\textbf{Алгоритм:}
\begin{enumerate}
    \item Выбираем начальную точку \( x_0 \) и заданную точность.
    \item Вычисляем новую точку, используя первую и вторую производные.
    \item Повторяем процесс, пока разница между последовательными значениями не станет меньше заданной точности.
\end{enumerate}

\textbf{Результат:}
\[ x_{\text{min}} \approx 0.649 \ldots, \quad f(x_{\text{min}}) \approx -1.957 \ldots \]

\section{Код программы на Python}

\textbf{Python-код для реализации методов средней точки, хорд и Ньютона:}

\begin{lstlisting}[language=Python]
import numpy as np
from scipy.optimize import minimize_scalar

def f(x):
    return np.cos(5 * x) - np.sin(2 * x)

def midpoint_method(a, b, tol=1e-5):
    while (b - a) > tol:
        midpoint = (a + b) / 2
        if f(midpoint - tol) < f(midpoint + tol):
            b = midpoint
        else:
            a = midpoint
    return (a + b) / 2


def secant_method(a, b, tol=1e-5):
    x0, x1 = a, b
    while abs(x1 - x0) > tol:
        f_x0, f_x1 = f(x0), f(x1)
        x2 = x1 - f_x1 * (x1 - x0) / (f_x1 - f_x0)
        x0, x1 = x1, x2
    return x1


def newton_method(x0, tol=1e-5):
    def df(x):
        return -5 * np.sin(5 * x) - 2 * np.cos(2 * x)
    
    def ddf(x):
        return -25 * np.cos(5 * x) + 4 * np.sin(2 * x)
    
    x = x0
    while abs(df(x)) > tol:
        x = x - df(x) / ddf(x)
    return x

Exc
a, b = 0, 1
x_min_midpoint = midpoint_method(a, b)
x_min_secant = secant_method(a, b)
x_min_newton = newton_method(0.5)

\end{lstlisting}

\section{Заключение}

В ходе работы были реализованы и протестированы три метода оптимизации для нахождения минимума функции \( f(x) = \cos(5x) - \sin(2x) \) на интервале \([0, 1]\). Все методы продемонстрировали близкие значения для минимума, однако метод Ньютона, как правило, сходится быстрее благодаря использованию второй производной. Полученные результаты подтверждают эффективность каждого из рассмотренных методов.

\section{Задача 2: Минимизация функции методом ломаных}

\textbf{Функция:}
\[
f(x) = (x + 1)(x - 1)(x - 3) - 1, \quad x \in [-2, 3]
\]

Цель задачи — найти минимум данной функции на интервале с помощью метода ломаных (broken line method).

\subsection{Метод ломаных}

Метод ломаных заключается в поэтапном нахождении точек на графике функции, между которыми соединяются прямые линии, и выбора подинтервала для следующей итерации на основе значений производной и функции.

\textbf{Алгоритм:}
\begin{enumerate}
    \item Инициализируем начальный интервал \( [a, b] \) и вычисляем \( L \), определяем начальные значения \( x \) и \( p \).
    \item В каждой итерации обновляем интервал и вычисляем новые точки, проверяя условие завершения.
    \item Повторяем процесс, пока условие завершения не будет выполнено.
\end{enumerate}

\begin{lstlisting}[language=Python]
    def f(x):
        return (x + 1)**2 * (x - 1) * (x - 3) - 1
    
    def diff(x):
        return 4 * x**3 - 6 * x**2 - 18 * x + 2
    
    def lomanye(a, b, epsilon):
        L = max(abs(diff(a)), abs(diff(b)))
        x = (1 / (2 * L)) * (f(a) - f(b) + L * (a + b))
        p = 0.5 * (f(a) + f(b) + L * (a - b))
        iter = False
        while True:
            if not iter:
                x0, p0 = x, p
            else:
                if p1 < p2:
                    x0, p0 = x1, p1
                else:
                    x0, p0 = x2, p2
            delta = (1 / (2 * L)) * (f(x0) - p0)
            sigma = 2 * L * delta
            if sigma <= epsilon:
                return x0, f(x0)
            x1 = x0 - delta
            x2 = x0 + delta
            p1 = 0.5 * (f(x1) + p0)
            p2 = 0.5 * (f(x2) + p0)
            iter = True
    
a, b = -2, 3
epsilon = 1e-5
x_star, f_star = lomanye(a, b, epsilon)
\end{lstlisting}

\textbf{Результат:}
\[ x_{\text{min}} \approx 2.28 \ldots, \quad f(x_{\text{min}}) \approx -10.91 \ldots \]

\end{document}