\documentclass[a4paper,12pt]{article}
\usepackage[T2A]{fontenc}
\usepackage[english]{babel} % языковой пакет
\usepackage{graphicx} % для картинок
\usepackage{amsmath,amsfonts,amssymb} %математика
\usepackage{mathtools}
\usepackage{pgfplots} % Для построения графиков
\pgfplotsset{compat=1.18} % Совместимость с новой версией pgfplots


\begin{document}
\section*{Задание 3. Лабораторная 1}

Убедимся в выпуклости функции \( f(x) \) на отрезке \( \left[ -\pi, \frac{3\pi}{2} \right] \).

Дана функция:
\[
f(x) = x^2 + \frac{x}{4} \cos x - 2
\]

1. Найдём первую производную \( f'(x) \):
\[
f'(x) = \frac{d}{dx} \left( x^2 + \frac{x}{4} \cos x - 2 \right)
\]
\[
f'(x) = 2x + \frac{1}{4} \cos x - \frac{x}{4} \sin x
\]

2. Найдём вторую производную \( f''(x) \):
\[
f''(x) = \frac{d}{dx} \left( 2x + \frac{1}{4} \cos x - \frac{x}{4} \sin x \right)
\]
\[
f''(x) = 2 - \frac{1}{4} \sin x - \frac{1}{4} \sin x - \frac{x}{4} \cos x
\]
\[
f''(x) = 2 - \frac{1}{2} \sin x - \frac{x}{4} \cos x
\]

3. Проверим знак второй производной на границах отрезка \( \left[ -\pi, \frac{3\pi}{2} \right] \).

Для \( x = -\pi \):
\[
f''(-\pi) = 2 - \frac{1}{2} \sin(-\pi) - \frac{-\pi}{4} \cos(-\pi)
\]
Так как \( \sin(-\pi) = 0 \) и \( \cos(-\pi) = -1 \), получаем:
\[
f''(-\pi) = 2 + \frac{\pi}{4}
\]

Для \( x = \frac{3\pi}{2} \):
\[
f''\left( \frac{3\pi}{2} \right) = 2 - \frac{1}{2} \sin\left( \frac{3\pi}{2} \right) - \frac{\frac{3\pi}{2}}{4} \cos\left( \frac{3\pi}{2} \right)
\]
Так как \( \sin\left( \frac{3\pi}{2} \right) = -1 \) и \( \cos\left( \frac{3\pi}{2} \right) = 0 \), то:
\[
f''\left( \frac{3\pi}{2} \right) = 2 + \frac{1}{2} = 2.5
\]

Таким образом, на границах отрезка вторая производная положительна.

4. Чтобы убедиться в выпуклости функции на всём отрезке \( \left[ -\pi, \frac{3\pi}{2} \right] \), необходимо построить график второй производной \( f''(x) \) или проанализировать её поведение аналитически. Если \( f''(x) \geq 0 \) на всём отрезке, то функция \( f(x) \) выпуклая.

3. Построим график функции \( f(x) \) на отрезке \( \left[ -\pi, \frac{3\pi}{2} \right] \).

\begin{center}
\begin{tikzpicture}
  \begin{axis}[
    domain=-pi:3*pi/2, % Интервал по x
    samples=100,
    axis x line=middle,
    axis y line=middle,
    xlabel={$x$},
    ylabel={$f(x)$},
    grid=both,
    width=10cm,
    height=6cm,
    title={График функции \( f(x) \)}
  ]
    \addplot[blue, thick] {x^2 + (x/4)*cos(deg(x)) - 2};
  \end{axis}
\end{tikzpicture}
\end{center}

4. Построим график второй производной \( f''(x) \) для проверки выпуклости.

\begin{center}
\begin{tikzpicture}
  \begin{axis}[
    domain=-pi:3*pi/2, % Интервал по x
    samples=100,
    axis x line=middle,
    axis y line=middle,
    xlabel={$x$},
    ylabel={$f''(x)$},
    grid=both,
    width=10cm,
    height=6cm,
    title={График второй производной \( f''(x) \)}
  ]
    \addplot[red, thick] {2 - (1/2)*sin(deg(x)) - (x/4)*cos(deg(x))};
  \end{axis}
\end{tikzpicture}
\end{center}

На графике видно, что вторая производная \( f''(x) \) неотрицательна на всём интервале, что подтверждает выпуклость функции \( f(x) \).

\end{document}