\documentclass[a4paper,12pt]{article}
\usepackage[utf8]{inputenc}
\usepackage[T2A]{fontenc}
\usepackage[english, russian]{babel}
\usepackage{amsmath}
\usepackage{amsfonts}
\usepackage{geometry}
\usepackage{graphicx}
\usepackage{xcolor}
\usepackage{listings}
\usepackage{fontspec}

\geometry{margin=1in}

% Настройки для кода в lstlisting
\lstset{
    inputencoding=utf8,
    extendedchars=\true,
    language=Python,
    basicstyle=\ttfamily\small,
    keywordstyle=\color{blue}\bfseries,
    commentstyle=\color{gray},
    stringstyle=\color{orange},
    showstringspaces=false,
    captionpos=b
}


% Установка шрифта
% \setmainfont{Times New Roman}

\begin{document}

% Центрированный текст на всю страницу
\begin{center}
    \textbf{\LARGE МИНОБРНАУКИ РОССИИ} \\[1em]
    \textbf{\large Федеральное государственное бюджетное образовательное учреждение} \\[1em]
    \textbf{\large высшего образования} \\[1em]
    \textbf{\large «САРАТОВСКИЙ НАЦИОНАЛЬНЫЙ ИССЛЕДОВАТЕЛЬСКИЙ} \\[1em]
    \textbf{\large ГОСУДАРСТВЕННЫЙ УНИВЕРСИТЕТ} \\[1em]
    \textbf{\large ИМЕНИ Н.Г. ЧЕРНЫШЕВСКОГО»} \\[2em]
    
    \textbf{\large Кафедра информатики и программирования} \\[2em]
    
    \textbf{\Large ЛАБОРАТОРНАЯ РАБОТА №2} \\[0.5em]
    
    \textbf{\large Вариант №24} \\[2em]
\end{center}

% Добавляем вертикальный отступ
\vfill

% Текст слева
\begin{flushleft}
    студента 3 курса 341 группы \\[0.5em]
    направления 02.03.03 – Математическое обеспечение и администрирование \\[1em]
    информационных систем \\[2em]
    
    факультета компьютерных наук и информационных технологий \\[2em]
    
    Филиппенко Дмитрия Александровича \\[2em]
\end{flushleft}

% Текст справа
\begin{flushright}
    к.ф.-м.н., доцент \\[1em]
    
    Рогачко Е. С. \\[2em]
\end{flushright}

% Центрированный текст внизу
\begin{center}
    Саратов 2024
\end{center}


% \maketitle
\newpage

\section*{Введение}
В данной работе рассматриваются методы оптимизации для нахождения минимума функций. Применяются следующие методы:
\begin{itemize}
    \item Метод поразрядного поиска
    \item Метод дихотомии
    \item Метод золотого сечения
    \item Метод параболической аппроксимации
\end{itemize}

\section*{Задание 1}
\subsection*{Постановка задачи}
Требуется найти минимум функции \( f(x) = 3x^2 - 2x - 2 \) на интервале \( x \in [-1, 1] \) с использованием методов поразрядного поиска, дихотомии и золотого сечения.

\subsection*{Метод поразрядного поиска}
Метод поразрядного поиска последовательно уменьшает шаг и меняет направление поиска, чтобы найти точку минимума функции.

\begin{lstlisting}[language=Python, caption=Метод поразрядного поиска]
import math

def f(x):
    return 3 * x ** 2 - 2 * x - 2

a, b = -1, 1
eps = 10 ** (-9)
delta = (b - a) / 4  # начальный шаг дискретизации
x0 = a
f0 = f(x0)

while math.fabs(delta) > eps:  # пока шаг дискретизации больше эпсилон
    x1 = x0 + delta
    f1 = f(x1)
    if f0 <= f1 or a >= x0 or x0 >= b:
        delta *= -1 / 4  # уменьшаем шаг и меняем направление
    x0, f0 = x1, f1

print(f'x = {x0}\nf = {f0}')
\end{lstlisting}

\paragraph{Результат работы:}
\[ 
x = 0.3333333320915699, \quad f(x) = -2.3333333333333335 
\]

\subsection*{Метод дихотомии}
Метод дихотомии использует две пробные точки, близкие к середине интервала, чтобы последовательно сужать интервал поиска.

\begin{lstlisting}[language=Python, caption=Метод дихотомии]
def f(x):
    return 3 * x ** 2 - 2 * x - 2

a, b = -1, 1
eps = 10 ** (-9)
eps_n = 1
delta = eps

while eps_n > eps:
    x1 = (a + b - delta) / 2
    x2 = (a + b + delta) / 2
    if f(x1) <= f(x2):
        b = x2  # сужаем интервал до [a, x2]
    else:
        a = x1  # сужаем интервал до [x1, b]
    eps_n = (b - a) / 2  # достигнутая точность

x_min = (a + b) / 2
f_min = f(x_min)
print(f'x = {x_min}\nf = {f_min}')
\end{lstlisting}

\paragraph{Результат работы:}
\[ 
x = 0.33333328303046816, \quad f(x) = -2.333333333333326 
\]

\subsection*{Метод золотого сечения}
Метод золотого сечения применяет пропорции золотого сечения для поиска минимального значения функции.

\begin{lstlisting}[language=Python, caption=Метод золотого сечения]
def f(x):
    return 3 * x ** 2 - 2 * x - 2

a, b = -1, 1
eps = 10 ** (-6)
tau = (5 ** 0.5 - 1) / 2  # коэффициент золотого сечения

x1 = a + (3 - 5 ** 0.5) / 2 * (b - a)
x2 = a + (5 ** 0.5 - 1) / 2 * (b - a)
f1 = f(x1)
f2 = f(x2)
eps_n = (b - a) / 2

while eps_n > eps:
    if f1 <= f2:
        b = x2
        x2 = x1
        f2 = f1
        x1 = b - tau * (b - a)
        f1 = f(x1)
    else:
        a = x1
        x1 = x2
        f1 = f2
        x2 = a + tau * (b - a)
        f2 = f(x2)
    eps_n *= tau

x_min = (a + b) / 2
f_min = f(x_min)
print(f'x = {x_min}\nf = {f_min}')
\end{lstlisting}

\paragraph{Результат работы:}
\[ 
x = 0.333333264898966, \quad f(x) = -2.3333333333333193 
\]

\section*{Задание 2}
\subsection*{Постановка задачи}
Найти минимум функции \( f(x) = 3x^3 + 6x^2 + x + 1 \) на интервале \( x \in [-3, 3] \) с использованием метода параболической аппроксимации.

\subsection*{Метод параболической аппроксимации}
Метод параболической аппроксимации строит параболу по трем точкам и находит вершину параболы как предполагаемый минимум функции.

\begin{lstlisting}[language=Python, caption=Метод параболической аппроксимации]
import math

def f(x):
    return 3 * x ** 3 + 6 * (x ** 2) + x + 1

a, b = -3, 3
eps = 10 ** (-6)
x1, x2, x3 = a, (a + b) / 2, b
f1, f2, f3 = f(x1), f(x2), f(x3)

a0 = f1
a1 = (f2 - f1) / (x2 - x1)
a2 = 1 / (x3 - x2) * ((f3 - f1) / (x3 - x1) - (f2 - f1) / (x2 - x1))
x_min = 1 / 2 * (x1 + x2 - a1 / a2)
f_min = f(x_min)
delta = 1

while math.fabs(delta) > eps:
    delta = x_min
    if f_min < f2:
        x1, x2, x3 = x2, x_min, x3
        f1, f2, f3 = f2, f_min, f3
    else:
        x1, x2, x3 = x1, x_min, x2
        f1, f2, f3 = f1, f_min, f2
    a0 = f1
    a1 = (f2 - f1) / (x2 - x1)
    a2 = 1 / (x3 - x2) * ((f3 - f1) / (x3 - x1) - (f2 - f1) / (x2 - x1))
    x_min = 1 / 2 * (x1 + x2 - a1 / a2)
    f_min = f(x_min)
    delta -= x_min

print(f'x = {x_min}\nf = {f_min}')
\end{lstlisting}

\paragraph{Результат работы:}
\[ 
x = -1.2440167337295778, \quad f(x) = 3.265811649490149 
\]

\section*{Заключение}
В данной работе исследованы методы оптимизации для поиска минимума функций. Проведенные эксперименты показали, что все методы успешно находят минимальные значения на заданных интервалах.

\end{document}
