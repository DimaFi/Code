\documentclass[a4paper,12pt]{article}
\usepackage[T2A]{fontenc}
\usepackage{amsmath}
\usepackage[english]{babel} % языковой пакет
\usepackage{graphicx}
\usepackage{geometry}
\geometry{margin=1in}
\usepackage{listings}

\title{Лабораторная 2}
\author{Филиппенко Дмитрий Александрович 341 группа}
\date{Вариант 24}

\begin{document}

\maketitle

\section{Задача 1: Минимизация квадратичной функции методами оптимизации}

\textbf{Функция:} 
\[
f(x) = 3x^2 - 2x - 2, \quad x \in [-1, 1]
\]

Цель задачи — найти минимум данной функции на отрезке с помощью методов: золотого сечения, дихотомии и поразрядного поиска.

\subsection{Метод золотого сечения}

Метод золотого сечения основан на делении интервала в пропорции золотого числа, которое примерно равно \(0.618\). На каждой итерации рассматриваются две точки внутри интервала, и в зависимости от значений функции в этих точках, интервал сужается до тех пор, пока его длина не станет меньше заданной точности.

\textbf{Алгоритм:}
\begin{enumerate}
    \item Инициализация: \( a = -1 \), \( b = 1 \), точность \( \text{tol} = 10^{-5} \).
    \item Вычисление коэффициента золотого сечения \( gr = \frac{\sqrt{5} - 1}{2} \).
    \item На каждой итерации вычисляются точки \( c \) и \( d \), которые делят интервал в пропорции золотого сечения.
    \item Сравнение значений функции в точках \( f(c) \) и \( f(d) \). В зависимости от этого сужаем интервал.
    \item Цикл продолжается до тех пор, пока длина интервала не станет меньше заданной точности.
\end{enumerate}

\textbf{Результат:}
\[
x_{\text{min}} \approx -0.3333, \quad f(x_{\text{min}}) \approx -2.3333
\]

\subsection{Метод дихотомии}

Метод дихотомии заключается в последовательном делении интервала на две части с помощью точек, расположенных симметрично относительно середины интервала. Интервал сужается на каждой итерации, пока его длина не станет меньше заданной точности.

\textbf{Алгоритм:}
\begin{enumerate}
    \item Инициализация: \( a = -1 \), \( b = 1 \), точность \( \text{tol} = 10^{-5} \), малое смещение \( \delta = 10^{-6} \).
    \item На каждой итерации вычисляются две точки \( x_1 = \frac{a + b}{2} - \delta \) и \( x_2 = \frac{a + b}{2} + \delta \).
    \item Сравниваем значения функции \( f(x_1) \) и \( f(x_2) \), сужаем интервал в зависимости от этих значений.
    \item Процесс повторяется, пока длина интервала не станет меньше заданной точности.
\end{enumerate}

\textbf{Результат:}
\[
x_{\text{min}} \approx 0.3333, \quad f(x_{\text{min}}) \approx -2.3333
\]

\subsection{Метод поразрядного поиска}

Метод поразрядного поиска основан на последовательном изменении значения переменной влево или вправо с фиксированным шагом, пока не будет найдено приближение минимума. Если при смещении в обе стороны не достигается улучшения, шаг уменьшается.

\textbf{Алгоритм:}
\begin{enumerate}
    \item Инициализация: начальная точка \( x_0 = 0 \), шаг \( \alpha = 0.1 \), точность \( \text{tol} = 10^{-5} \).
    \item На каждой итерации сравниваем значения функции в точках \( x - \text{step} \), \( x \) и \( x + \text{step} \).
    \item Если \( f(x - \text{step}) < f(x) \), смещаемся влево, если \( f(x + \text{step}) < f(x) \), смещаемся вправо.
    \item Если ни одно смещение не уменьшает значение функции, уменьшаем шаг вдвое.
\end{enumerate}

\textbf{Результат:}
\[
x_{\text{min}} \approx 0.3333, \quad f(x_{\text{min}}) \approx -2.3333
\]

\section{Задача 2: Минимизация кубической функции методом парабол}

\textbf{Функция:}
\[
f(x) = 3x^3 + 6x^2 + x + 1, \quad x \in [-3, 3]
\]

Цель задачи — найти минимум данной кубической функции на отрезке методом парабол.

\subsection{Метод парабол}

Метод парабол заключается в аппроксимации функции параболой через три точки на интервале и нахождении вершины этой параболы, которая является приближением минимума функции.

\textbf{Алгоритм:}
\begin{enumerate}
    \item Инициализация: \( a = -3 \), \( b = 3 \), точность \( \text{tol} = 10^{-5} \).
    \item Выбираем три точки: \( x_1 = a \), \( x_2 = \frac{a + b}{2} \), \( x_3 = b \).
    \item На каждой итерации вычисляются значения функции в точках \( f(x_1), f(x_2), f(x_3) \).
    \item Аппроксимируем параболу, вычисляем вершину параболы \( x_{\text{min}} \) с помощью формул для коэффициентов аппроксимации.
    \item Если значение функции в точке \( x_{\text{min}} \) меньше, чем в \( x_2 \), то сужаем интервал, выбирая новые точки.
    \item Процесс повторяется, пока длина интервала не станет меньше заданной точности.
\end{enumerate}

\textbf{Результат:}
\[
x_{\text{min}} \approx -1.244016, \quad f(x_{\text{min}}) \approx 3.265811649490149
\]

\section{Код программы на Python}

\textbf{Python-код для задачи 1 и задачи 2:}

\begin{lstlisting}[language=Python]
    import numpy as np
    
    # 1. First task: Methods of optimization (golden section, dichotomy, coordinate search)
    def f(x):
        return 3*x**2 - 2*x - 2
    
    # Method of golden section search
    def golden_section_search(a, b, tol=1e-5):
        gr = (np.sqrt(5) - 1) / 2
        c = b - gr * (b - a)
        d = a + gr * (b - a)
        
        while abs(b - a) > tol:
            if f(c) < f(d):
                b = d
            else:
                a = c
            
            c = b - gr * (b - a)
            d = a + gr * (b - a)
        
        return (b + a) / 2
    
    # Method of dichotomy search
    def dichotomy_search(a, b, tol=1e-5, delta=1e-6):
        while abs(b - a) > tol:
            x1 = (a + b) / 2 - delta
            x2 = (a + b) / 2 + delta
            
            if f(x1) < f(x2):
                b = x2
            else:
                a = x1
        
        return (a + b) / 2
    
    # Method of coordinate search
    def coordinate_search(x0, tol=1e-5, alpha=0.1):
        x = x0
        step = alpha
        while abs(step) > tol:
            f_left = f(x - step)
            f_right = f(x + step)
            
            if f_left < f(x):
                x = x - step
            elif f_right < f(x):
                x = x + step
            else:
                step *= 0.5
        
        return x
    
    # 2. Second task: Parabolic method
    
    def f_cubic(x):
        return 3*x**3 + 6*x**2 + x + 1
    
    # Method of parabolic search
    def parabola_search(a, b, tol=1e-5):
        x1, x2, x3 = a, (a + b) / 2, b
        
        while abs(b - a) > tol:
            f1, f2, f3 = f_cubic(x1), f_cubic(x2), f_cubic(x3)
            
            num = (x2**2 - x3**2)*f1 + (x3**2 - x1**2)*f2 + (x1**2 - x2**2)*f3
            denom = (x2 - x3)*f1 + (x3 - x1)*f2 + (x1 - x2)*f3
            
            x_min = 0.5 * num / denom
            
            if f_cubic(x_min) < f_cubic(x2):
                if x_min < x2:
                    x3 = x2
                else:
                    x1 = x2
                x2 = x_min
            else:
                if x_min < x2:
                    x1 = x_min
                else:
                    x3 = x_min
        
        return x_min
    
    # Running the methods
    if __name__ == '__main__':
        # Task 1: Optimization methods
        
        a, b = -1, 1
        
        # Golden section search
        x_min_golden = golden_section_search(a, b)
        print(f"Golden section method: x_min = {x_min_golden}, f(x_min) = {f(x_min_golden)}")
        
        # Dichotomy search
        x_min_dichotomy = dichotomy_search(a, b)
        print(f"Dichotomy method: x_min = {x_min_dichotomy}, f(x_min) = {f(x_min_dichotomy)}")
        
        # Coordinate search
        x0 = 0
        x_min_coordinate = coordinate_search(x0)
        print(f"Coordinate search method: x_min = {x_min_coordinate}, f(x_min) = {f(x_min_coordinate)}")
        
        # Task 2: Parabolic method for cubic function
        
        a, b = -3, 3
        
        # Parabolic search
        x_min_parabola = parabola_search(a, b)
        print(f"Parabolic method: x_min = {x_min_parabola}, f(x_min) = {f_cubic(x_min_parabola)}")
    \end{lstlisting}
    
    \end{document}