\documentclass[a4paper,12pt]{article}
\usepackage[T2A]{fontenc}
\usepackage[english]{babel} % языковой пакет
\usepackage{graphicx} % для картинок
\usepackage{amsmath,amsfonts,amssymb} %математика
\usepackage{mathtools}
\usepackage{pgfplots} % Для построения графиков
\pgfplotsset{compat=1.18} % Совместимость с новой версией pgfplots

\begin{document}

\section*{Задача 4 Лабортаторная 1: Найти наименьшую константу Липшица функции \( f(x) \) на заданных отрезках}

Дана функция:
\[
f(x) = \frac{x^4}{20} (3x - 40) + x(6x^2 + 1)
\]
Найдём её константы Липшица на двух отрезках:
\[
a) \quad [0, 2], \quad b) \quad [2, 6]
\]

\section*{1. Найдём первую производную \( f'(x) \)}

Используем правило дифференцирования суммы и произведения. 

1.1. Первая часть \( \frac{x^4}{20}(3x - 40) \):
Рассмотрим отдельно произведение \( \frac{x^4}{20} \cdot (3x - 40) \). Используем правило произведения:
\[
f_1(x) = \frac{x^4}{20}, \quad g_1(x) = 3x - 40
\]
Производная будет:
\[
f_1'(x) = \frac{4x^3}{20} = \frac{x^3}{5}, \quad g_1'(x) = 3
\]
Применяя правило произведения:
\[
\frac{d}{dx} \left( \frac{x^4}{20}(3x - 40) \right) = \frac{x^3}{5}(3x - 40) + \frac{3x^4}{20}
\]

1.2. Вторая часть \( x(6x^2 + 1) \):
Теперь дифференцируем \( x(6x^2 + 1) \):
\[
f_2(x) = x, \quad g_2(x) = 6x^2 + 1
\]
Производная будет:
\[
f_2'(x)g_2(x) + f_2(x)g_2'(x) = (6x^2 + 1) + x \cdot 12x = 6x^2 + 1 + 12x^2 = 18x^2 + 1
\]

1.3. Полная производная \( f'(x) \):
Теперь соберём полную производную \( f'(x) \):
\[
f'(x) = \frac{x^3}{5}(3x - 40) + \frac{3x^4}{20} + 18x^2 + 1
\]
Упрощаем выражение:
\[
f'(x) = \frac{3x^4}{20} - \frac{8x^3}{5} + 18x^2 + 1
\]

\section*{2. Поиск константы Липшица на отрезках}

Константа Липшица \( L \) на отрезке — это максимальное значение абсолютной величины первой производной на отрезке:
\[
L = \sup_{x_1, x_2 \in [a, b]} \frac{|f'(x_1) - f'(x_2)|}{|x_1 - x_2|}
\]
Найдем максимумы \( f'(x) \) на каждом отрезке.

2.1. Оценка на отрезке \( [0, 2] \):
Подставим значения из отрезка в производную и найдём экстремумы:
- Для \( x = 0 \):
\[
f'(0) = \frac{3(0)^4}{20} - \frac{8(0)^3}{5} + 18(0)^2 + 1 = 1
\]
- Для \( x = 2 \):
\[
f'(2) = \frac{3(2)^4}{20} - \frac{8(2)^3}{5} + 18(2)^2 + 1 = \frac{48}{20} - \frac{64}{5} + 72 + 1 = 2.4 - 12.8 + 72 + 1 = 62.6
\]

Теперь найдём критические точки, исследуя производную:
\[
f'(x) = 0 \quad \Rightarrow \quad \frac{3x^4}{20} - \frac{8x^3}{5} + 18x^2 + 1 = 0
\]
Для решения этого уравнения можно использовать численные методы.

2.2. Оценка на отрезке \( [2, 6] \):
Аналогично, подставим значения из отрезка \( [2, 6] \) в производную и найдём экстремумы:
- Для \( x = 6 \):
\[
f'(6) = \frac{3(6)^4}{20} - \frac{8(6)^3}{5} + 18(6)^2 + 1 = 388.8
\]

2.3. Построим график первой производной для наглядности

\begin{center}
\begin{tikzpicture}
  \begin{axis}[
    domain=0:6, % Интервал по x
    samples=100,
    axis x line=middle,
    axis y line=middle,
    xlabel={$x$},
    ylabel={$f'(x)$},
    grid=both,
    width=10cm,
    height=6cm,
    title={График первой производной \( f'(x) \)}
  ]
    \addplot[blue, thick] {(3*x^4)/20 - (8*x^3)/5 + 18*x^2 + 1};
  \end{axis}
\end{tikzpicture}
\end{center}

\section*{3. Результаты}

Константа Липшица на отрезке \( [0, 2] \) равна \( L_1 = 62.6 \), а на отрезке \( [2, 6] \) — \( L_2 = 388.8 \). Минимальная константа Липшица — это \( L_1 = 62.6 \).

\end{document}