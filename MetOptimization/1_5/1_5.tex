\documentclass[a4paper,12pt]{article}
\usepackage[T2A]{fontenc}
\usepackage[english]{babel} % языковой пакет
\usepackage{graphicx} % для картинок
\usepackage{amsmath,amsfonts,amssymb} %математика
\usepackage{mathtools}
\usepackage{pgfplots} % Для построения графиков
\pgfplotsset{compat=1.18} % Совместимость с новой версией pgfplots

\begin{document}

\section*{Задача 5 Лабораторная 1: Найти минимум функции \( f(x) = \frac{x^3}{x^2 - 1} \) на заданных интервалах}

Функция \( f(x) = \frac{x^3}{x^2 - 1} \) определена на отрезках \( \left[ -3, -\frac{3}{2} \right] \) и \( \left[ \frac{3}{2}, 2 \right] \). 

1. Найдём производную функции \( f'(x) \)

Производная функции \( f(x) = \frac{x^3}{x^2 - 1} \) находится с использованием правила Лейбница для дифференцирования частного:
\[
f'(x) = \frac{(x^2 - 1) \cdot 3x^2 - x^3 \cdot 2x}{(x^2 - 1)^2}
\]
Рассчитаем числитель:
\[
f'(x) = \frac{3x^2(x^2 - 1) - 2x^4}{(x^2 - 1)^2} = \frac{3x^4 - 3x^2 - 2x^4}{(x^2 - 1)^2} = \frac{x^4 - 3x^2}{(x^2 - 1)^2}
\]
Таким образом:
\[
f'(x) = \frac{x^2(x^2 - 3)}{(x^2 - 1)^2}
\]

2. Найдём критические точки

Критические точки находятся из уравнения \( f'(x) = 0 \). Решим уравнение:
\[
\frac{x^2(x^2 - 3)}{(x^2 - 1)^2} = 0
\]
Числитель \( x^2(x^2 - 3) = 0 \), откуда \( x^2 = 0 \) или \( x^2 = 3 \).

- \( x = 0 \) — не принадлежит заданным интервалам, поэтому его не рассматриваем.
- \( x = \pm \sqrt{3} \approx \pm 1.732 \).

3. Поведение функции на заданных интервалах

3.1. Интервал \( \left[ -3, -\frac{3}{2} \right] \)

На этом интервале рассматриваем значения функции в краевых точках и в точке \( x = -\sqrt{3} \approx -1.732 \):
- Для \( x = -3 \):
\[
f(-3) = \frac{(-3)^3}{(-3)^2 - 1} = \frac{-27}{9 - 1} = \frac{-27}{8} = -3.375
\]
- Для \( x = -\frac{3}{2} \):
\[
f\left( -\frac{3}{2} \right) = \frac{\left( -\frac{3}{2} \right)^3}{\left( -\frac{3}{2} \right)^2 - 1} = \frac{-\frac{27}{8}}{\frac{9}{4} - 1} = \frac{-\frac{27}{8}}{\frac{5}{4}} = -\frac{27}{10} = -2.7
\]
- Для \( x = -\sqrt{3} \approx -1.732 \):
\[
f(-\sqrt{3}) = \frac{(-\sqrt{3})^3}{(-\sqrt{3})^2 - 1} = \frac{-3\sqrt{3}}{3 - 1} = \frac{-3\sqrt{3}}{2} \approx -2.598
\]

Таким образом, на интервале \( \left[ -3, -\frac{3}{2} \right] \) минимум достигается в точке \( x = -3 \) и равен \( f(-3) = -3.375 \).

3.2. Интервал \( \left[ \frac{3}{2}, 2 \right] \)

Теперь исследуем функцию на интервале \( \left[ \frac{3}{2}, 2 \right] \):
- Для \( x = \frac{3}{2} \):
\[
f\left( \frac{3}{2} \right) = \frac{\left( \frac{3}{2} \right)^3}{\left( \frac{3}{2} \right)^2 - 1} = \frac{\frac{27}{8}}{\frac{9}{4} - 1} = \frac{\frac{27}{8}}{\frac{5}{4}} = \frac{27}{10} = 2.7
\]
- Для \( x = 2 \):
\[
f(2) = \frac{2^3}{2^2 - 1} = \frac{8}{4 - 1} = \frac{8}{3} \approx 2.667
\]
- Для \( x = \sqrt{3} \approx 1.732 \):
\[
f(\sqrt{3}) = \frac{(\sqrt{3})^3}{(\sqrt{3})^2 - 1} = \frac{3\sqrt{3}}{3 - 1} = \frac{3\sqrt{3}}{2} \approx 2.598
\]

Таким образом, на интервале \( \left[ \frac{3}{2}, 2 \right] \) минимум достигается в точке \( x = \sqrt{3} \approx 1.732 \) и равен \( f(\sqrt{3}) \approx 2.598 \).

4. График функции \( f(x) \)

Для наглядности построим график функции \( f(x) = \frac{x^3}{x^2 - 1} \) на двух интервалах.

\begin{center}
\begin{tikzpicture}
  \begin{axis}[
    domain=-3:2, % Интервал по x
    samples=200,
    axis x line=middle,
    axis y line=middle,
    xlabel={$x$},
    ylabel={$f(x)$},
    grid=both,
    width=10cm,
    height=8cm,
    title={График функции \( f(x) = \frac{x^3}{x^2 - 1} \)}
  ]
    \addplot[blue, thick] {(x^3)/(x^2 - 1)};
    \addlegendentry{$f(x) = \frac{x^3}{x^2 - 1}$}
  \end{axis}
\end{tikzpicture}
\end{center}

\section*{5. Результаты}

- На интервале \( \left[ -3, -\frac{3}{2} \right] \) минимум функции равен \( f(-3) = -3.375 \).
- На интервале \( \left[ \frac{3}{2}, 2 \right] \) минимум функции равен \( f(\sqrt{3}) \approx 2.598 \).

\end{document}
