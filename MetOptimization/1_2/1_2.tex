\documentclass[a4paper,12pt]{article}
\usepackage[T2A]{fontenc}
\usepackage[english]{babel} % языковой пакет
\usepackage{graphicx} % для картинок
\usepackage{amsmath,amsfonts,amssymb} %математика
\usepackage{mathtools}
\usepackage{pgfplots} % Для построения графиков
\pgfplotsset{compat=1.17} % Совместимость с новой версией pgfplots

\begin{document}

\section*{Задание 2. Лабораторная 1}

Рассмотрим функцию:
\[
f(x) = x^2 - 5x(x - 4) - 7.
\]
Раскроем скобки:
\[
f(x) = x^2 - 5x^2 + 20x - 7 = -4x^2 + 20x - 7.
\]
Это квадратичная функция вида \( f(x) = ax^2 + bx + c \), где:
\[
a = -4, \, b = 20, \, c = -7.
\]

\subsection*{Унимодальность функции}
Квадратичная функция является унимодальной, то есть она имеет либо максимум, либо минимум. Поскольку коэффициент \( a = -4 \) отрицателен, парабола направлена вниз, и функция имеет максимум в вершине.

Найдём координаты вершины параболы:
\[
x_{\text{max}} = \frac{-b}{2a} = \frac{-20}{2(-4)} = 2.5.
\]
Следовательно, максимум функции достигается при \( x = 2.5 \).

Для того чтобы функция оставалась унимодальной на отрезке \( [-2, b] \), максимальное значение \( b \), при котором это условие выполняется, равно \( 2.5 \). Таким образом, на отрезке \( [-2, 2.5] \) функция унимодальна.

\section*{График функции}

\begin{center}
\begin{tikzpicture}
    \begin{axis}[
        domain=-5:7,
        samples=400,
        axis lines=middle,
        xlabel=$x$, ylabel={$f(x)$},
        grid=major,
        width=10cm, height=7cm,
        title={График функции $f(x) = -4x^2 + 20x - 7$},
        legend style={at={(1.05,1)}, anchor=north west}
    ]
    % Plot the function
    \addplot[
        thick, blue,
    ] {-4*x^2 + 20*x - 7};
    \addlegendentry{$f(x) = -4x^2 + 20x - 7$}
    
    % Vertical lines for the interval [-2, 2.5]
    \addplot[dashed, thick, green] coordinates {(-2, -50) (-2, 50)};
    \addlegendentry{$x = -2$}
    
    \addplot[dashed, thick, purple] coordinates {(2.5, -50) (2.5, 50)};
    \addlegendentry{$x = 2.5$}
    
    % Maximum point
    \addplot[mark=*, red] coordinates {(2.5, 18.75)};
    \addlegendentry{Максимум при $x = 2.5$}
    
    \end{axis}
\end{tikzpicture}
\end{center}

\end{document}
