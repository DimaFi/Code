\documentclass[a4paper,12pt]{article}
\usepackage[utf8]{inputenc}
\usepackage[russian]{babel}
\usepackage{amsmath, amssymb, graphicx, geometry}
\geometry{left=2.5cm,right=2.5cm,top=2.5cm,bottom=2.5cm}


\begin{document}

\section*{Решение задачи линейного программирования}

\subsection*{Формулировка задачи}
Найти максимум функции 
\[
z = 3x_1 - x_2,
\]
при следующих ограничениях:
\[
\begin{aligned}
    &2x_1 - x_2 \leq 4, \\
    &x_1 - 2x_2 \leq 2, \\
    &x_1 + x_2 \leq 5, \\
    &x_1 \geq 0, \quad x_2 \geq 0.
\end{aligned}
\]

\subsection*{Теория линейного программирования}
Линейное программирование — это метод оптимизации, который используется для нахождения максимального или минимального значения линейной функции при наличии системы линейных ограничений. В рамках линейного программирования решается задача в следующей общей форме:
\begin{itemize}
    \item \textbf{Целевая функция:} $z = c_1x_1 + c_2x_2 + \dots + c_nx_n$, где $c_i$ — коэффициенты целевой функции.
    \item \textbf{Ограничения:} система линейных неравенств, задающая допустимую область решений.
    \item \textbf{Допустимая область:} пересечение полуплоскостей, задаваемых ограничениями, обычно образует выпуклый многогранник.
\end{itemize}
Максимум или минимум целевой функции достигается в одной из вершин допустимой области.

\subsection*{Графическое решение задачи}
1. \textbf{Построение ограничений.} Выразим каждое ограничение в виде равенств для построения прямых:
\[
\begin{aligned}
    &2x_1 - x_2 = 4 \quad \Rightarrow \quad x_2 = 2x_1 - 4, \\
    &x_1 - 2x_2 = 2 \quad \Rightarrow \quad x_2 = \frac{x_1 - 2}{2}, \\
    &x_1 + x_2 = 5 \quad \Rightarrow \quad x_2 = 5 - x_1.
\end{aligned}
\]
Кроме того, учтём ограничения $x_1 \geq 0$ и $x_2 \geq 0$, что означает, что область решений находится в первом квадранте.

2. \textbf{Определение области допустимых решений.} Найдём точки пересечения прямых:
\begin{itemize}
    \item Пересечение $2x_1 - x_2 = 4$ и $x_1 - 2x_2 = 2$:
    \[
    \begin{aligned}
        &2x_1 - x_2 = 4, \\
        &x_1 - 2x_2 = 2.
    \end{aligned}
    \]
    Решая систему, получаем точку $(x_1, x_2) = (2, 0)$.
    \item Пересечение $2x_1 - x_2 = 4$ и $x_1 + x_2 = 5$:
    \[
    \begin{aligned}
        &2x_1 - x_2 = 4, \\
        &x_1 + x_2 = 5.
    \end{aligned}
    \]
    Решая систему, получаем точку $(x_1, x_2) = (3, 2)$.
    \item Пересечение $x_1 - 2x_2 = 2$ и $x_1 + x_2 = 5$:
    \[
    \begin{aligned}
        &x_1 - 2x_2 = 2, \\
        &x_1 + x_2 = 5.
    \end{aligned}
    \]
    Решая систему, получаем точку $(x_1, x_2) = (4, 1)$.
\end{itemize}
Таким образом, вершины области допустимых решений: $(0, 0)$, $(2, 0)$, $(3, 2)$, $(4, 1)$. 

3. \textbf{Вычисление значений целевой функции.} Подставляем координаты вершин в целевую функцию:
\[
\begin{aligned}
    &z(0, 0) = 3 \cdot 0 - 0 = 0, \\
    &z(2, 0) = 3 \cdot 2 - 0 = 6, \\
    &z(3, 2) = 3 \cdot 3 - 2 = 7, \\
    &z(4, 1) = 3 \cdot 4 - 1 = 11.
\end{aligned}
\]

4. \textbf{Оптимальное решение.} Максимальное значение функции $z = 11$ достигается в точке $(4, 1)$.

\end{document}