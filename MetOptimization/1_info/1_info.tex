\documentclass[a4paper,12pt]{article}
\usepackage[T2A]{fontenc}
\usepackage[english]{babel} % языковой пакет
\usepackage{graphicx} % для картинок
\usepackage{amsmath,amsfonts,amssymb} %математика
\usepackage{mathtools}
\usepackage{pgfplots} % Для построения графиков
\pgfplotsset{compat=1.18} % Совместимость с новой версией pgfplots

\begin{document}

\title{Памятка по математическому анализу}
\author{}
\date{}
\maketitle

\section*{1. Производные и нахождение экстремумов функций}

Производная функции \( f(x) \) показывает скорость изменения функции относительно изменения переменной \( x \). Она используется для нахождения \textbf{критических точек}, где производная равна нулю или не существует, что помогает искать экстремумы (минимумы или максимумы) функции.

\subsection*{Алгоритм поиска экстремумов:}
\begin{enumerate}
    \item Найдите производную \( f'(x) \) функции \( f(x) \).
    \item Решите уравнение \( f'(x) = 0 \) для поиска критических точек.
    \item Определите знак производной на интервалах, на которые критические точки разбивают область определения функции. Это поможет понять, где функция возрастает (\( f'(x) > 0 \)) или убывает (\( f'(x) < 0 \)).
    \item Проверьте краевые точки: если функция изучается на отрезке \( [a, b] \), необходимо вычислить значение функции в концах отрезка.
\end{enumerate}

\section*{2. Выпуклость функции}

Функция выпуклая на промежутке, если касательные к графику функции на этом промежутке лежат \textbf{ниже} самого графика. Если касательные лежат \textbf{выше}, функция называется вогнутой. Для проверки выпуклости используется \textbf{вторая производная}.

\subsection*{Алгоритм проверки выпуклости:}
\begin{enumerate}
    \item Найдите вторую производную \( f''(x) \).
    \item Если \( f''(x) > 0 \) на всём промежутке, то функция \textbf{выпуклая}.
    \item Если \( f''(x) < 0 \) на всём промежутке, то функция \textbf{вогнутая}.
    \item Если знак второй производной меняется, то нужно провести более детальный анализ изменения выпуклости.
\end{enumerate}

\section*{3. Константа Липшица}

Константа Липшица определяет, насколько сильно может изменяться значение функции при небольшом изменении аргумента. Если функция удовлетворяет \textbf{условию Липшица}, то она не изменяется быстрее, чем это определяет константа Липшица.

\subsection*{Условие Липшица:}
\[
|f(x_1) - f(x_2)| \leq L |x_1 - x_2|
\]
где \( L \) — это константа Липшица.

\subsection*{Алгоритм нахождения константы Липшица:}
\begin{enumerate}
    \item Найдите первую производную \( f'(x) \) функции.
    \item Определите максимальное значение \( |f'(x)| \) на заданном интервале. Это значение и будет константой Липшица.
\end{enumerate}

\section*{4. Правила Лейбница для производных}

Правило Лейбница позволяет вычислять производные сложных функций, таких как произведение или частное двух функций.

\subsection*{Производная произведения функций:}
\[
(f \cdot g)' = f' \cdot g + f \cdot g'
\]

\subsection*{Производная частного функций:}
\[
\left( \frac{f}{g} \right)' = \frac{f' \cdot g - f \cdot g'}{g^2}
\]

Это правило часто используется при нахождении производных функций вида \( \frac{x^3}{x^2 - 1} \) или других дробных выражений.

\section*{5. Анализ на интервалах и критические точки}

При нахождении экстремумов на интервалах важно учитывать не только критические точки внутри интервала, но и \textbf{пограничные точки}.

\begin{enumerate}
    \item Для критических точек \( f'(x) = 0 \) анализируйте изменение знака производной:
    \begin{itemize}
        \item Если \( f'(x) \) меняет знак с положительного на отрицательный, в точке будет \textbf{максимум}.
        \item Если \( f'(x) \) меняет знак с отрицательного на положительный, в точке будет \textbf{минимум}.
    \end{itemize}
    \item Не забывайте про краевые точки интервала. Минимум или максимум функции могут находиться в них.
\end{enumerate}

\section*{6. Графики функций и интерпретация}

Графики помогают визуализировать поведение функции на различных интервалах, что полезно при анализе выпуклости, экстремумов и общей динамики функции.

\subsection*{Построение графика:}
\begin{itemize}
    \item Для построения графика функции важно учитывать критические точки, интервалы возрастания/убывания и выпуклость.
    \item Используйте программы для построения графиков, такие как Python, LaTeX (с библиотекой \texttt{pgfplots}), или онлайн инструменты.
\end{itemize}

\section*{Пример применения}

Для функции \( f(x) = \frac{x^3}{x^2 - 1} \):
\begin{enumerate}
    \item Найдена производная \( f'(x) = \frac{x^2(x^2 - 3)}{(x^2 - 1)^2} \).
    \item Решено уравнение \( f'(x) = 0 \), найдены критические точки \( x = \pm \sqrt{3} \).
    \item Проанализировано поведение функции в критических точках и на границах интервалов \( \left[ -3, -\frac{3}{2} \right] \) и \( \left[ \frac{3}{2}, 2 \right] \).
    \item Построен график функции для визуального анализа.
\end{enumerate}

\section*{Полезные источники:}
\begin{itemize}
    \item Учебники по математическому анализу (например, Демидович, Зорич).
    \item Онлайн-курсы на платформах, таких как \href{https://www.coursera.org/}{Coursera}, \href{https://www.khanacademy.org/}{Khan Academy}.
    \item Системы компьютерной алгебры (Mathematica, Maple, SymPy в Python) для проверки производных и построения графиков.
\end{itemize}


\section*{Памятка по основным темам}

\subsection*{1. Целевая функция}
\textbf{Целевая функция} — это функция, которую необходимо минимизировать или максимизировать в задаче оптимизации. В различных прикладных задачах целевая функция может представлять затраты, прибыль, потери, выигрыш и т.д. 

\subsection*{2. Локальный и глобальный минимум функции}
- \textbf{Локальный минимум} функции \( f(x) \) в точке \( x_0 \) — это такая точка, что \( f(x_0) \leq f(x) \) для всех \( x \) в некоторой окрестности точки \( x_0 \). Это означает, что вблизи \( x_0 \) функция не принимает меньших значений.
- \textbf{Глобальный минимум} функции \( f(x) \) на множестве \( A \) — это такая точка \( x_0 \), что \( f(x_0) \leq f(x) \) для всех \( x \in A \). Таким образом, глобальный минимум является наименьшим значением функции на всём множестве.

\subsection*{3. Точная нижняя грань функции на множестве}
\textbf{Точная нижняя грань (инфимум)} функции на множестве — это наибольшее число, которое меньше либо равно всем значениям функции на данном множестве. Если функция достигает своей точной нижней грани в какой-то точке множества, то эта точка является минимумом функции.

\subsection*{4. Соотношение между точной нижней гранью и минимумом функции}
Точная нижняя грань функции на множестве может совпадать с минимумом функции, если функция достигает этого значения. В противном случае точная нижняя грань может быть меньше значения функции в любой точке множества, если минимум не достигается.

\subsection*{5. Унимодальная функция на отрезке}
\textbf{Унимодальная функция} — это функция, которая на отрезке \( [a, b] \) имеет не более одной точки экстремума (максимума или минимума). Она либо возрастает до точки экстремума, а затем убывает, либо убывает до этой точки, а затем возрастает.

\subsection*{6. Свойства унимодальных функций}
- Унимодальная функция имеет одно экстремальное значение на всём отрезке \( [a, b] \).
- Если функция убывает на одном участке и возрастает на другом, то экстремум является минимумом.
- Если функция возрастает, а затем убывает, то экстремум является максимумом.

\subsection*{7. Выпуклая функция на отрезке \( [a, b] \)}
\textbf{Выпуклая функция} на отрезке \( [a, b] \) — это функция, график которой лежит ниже любой прямой, соединяющей две произвольные точки на графике функции.

\subsection*{8. Геометрический смысл выпуклой функции}
Геометрический смысл выпуклой функции заключается в том, что отрезки между любыми двумя точками на её графике не пересекают график функции. То есть функция образует «чашеобразную» форму, если она выпуклая, и «куполообразную», если вогнутая.

\subsection*{9. Необходимые и достаточные дифференциальные условия выпуклости}
Для дважды дифференцируемой функции \( f(x) \) на отрезке \( [a, b] \):
- Функция выпуклая на \( [a, b] \), если её вторая производная \( f''(x) \geq 0 \) для всех \( x \in [a, b] \).
- Функция вогнутая на \( [a, b] \), если \( f''(x) \leq 0 \).

\subsection*{10. Условие Липшица для функции на отрезке}
Функция \( f(x) \) удовлетворяет \textbf{условию Липшица} на отрезке \( [a, b] \), если существует постоянная \( L \geq 0 \), такая что для любых \( x_1, x_2 \in [a, b] \) выполняется неравенство:
\[
|f(x_1) - f(x_2)| \leq L |x_1 - x_2|.
\]
Это означает, что функция \( f(x) \) имеет ограниченную скорость изменения на отрезке \( [a, b] \).

\subsection*{11. Всякая ли унимодальная функция удовлетворяет условию Липшица?}
Не всякая унимодальная функция на отрезке удовлетворяет условию Липшица. Унимодальная функция может иметь резкие скачки или сильные изменения в поведении, что нарушает ограниченность скорости изменения функции, требуемую по условию Липшица.

\subsection*{12. Всякая ли функция, удовлетворяющая условию Липшица, унимодальна?}
Не всякая функция, удовлетворяющая условию Липшица, является унимодальной. Липшицевы функции могут иметь более одного экстремума на заданном отрезке, что нарушает унимодальность.

\subsection*{13. Свойства функций, удовлетворяющих условию Липшица}
- Липшицевы функции равномерно непрерывны на отрезке.
- Скорость изменения функции ограничена постоянной \( L \).
- Липшицевы функции не имеют «бесконечных» резких скачков.

\subsection*{14. Классический метод минимизации функций}
Классический метод минимизации функций заключается в нахождении критических точек, где производная функции равна нулю (\( f'(x) = 0 \)). После этого проводится проверка второго порядка (вторая производная функции), чтобы определить, является ли эта точка минимумом, максимумом или точкой перегиба:
- Если \( f''(x) > 0 \), точка является минимумом.
- Если \( f''(x) < 0 \), точка является максимумом.
- Если \( f''(x) = 0 \), необходимо дополнительное исследование для определения характера точки.

\subsection*{Применение к задачам}

\subsection*{Задание 1: Унимодальность функции \( f(x) = 3x - x|x - 6| + 5 \)}
Функция разбивается на два случая:
\[
f(x) = 
\begin{cases} 
-x^2 + 9x + 5, & \text{если } x \geq 6, \\
x^2 - 3x + 5, & \text{если } x < 6.
\end{cases}
\]
На отрезках \( [1, 6) \) и \( [6, 10] \) функция является унимодальной.

\subsection*{Задание 2: Поиск максимума функции \( f(x) = -4x^2 + 20x - 7 \)}
Квадратичная функция \( f(x) = -4x^2 + 20x - 7 \) имеет вершину в точке:
\[
x_{\text{max}} = \frac{-b}{2a} = \frac{-20}{2(-4)} = 2.5.
\]
Максимум функции достигается при \( x = 2.5 \), и она унимодальна на отрезке \( [-2, 2.5] \).

\end{document}